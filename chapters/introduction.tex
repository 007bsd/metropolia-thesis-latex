% Introduction

\chapter{Introduction}

Embedded systems are growing rapidly and becoming really important component of modern life. They
are deployed in devices such as workstations, microprocessors, servers including different consumer
electronics devices, cell phones, interconnected \gls{iot} devices, medical instruments,
autonomous and connected vehicles and many more\cite{camposano1996embedded:ARTICLE}. In a recent study
shows\cite{IoTconne16:online} that the number of such devices are estimated to grow more than
29 billion devices by the year 2030 due to the rapid rise in day-to-day use. Due to this growth, the
amount of related security vulnerabilities are also increasing. The recent Mirai malware attacks,
third party remote exploitation on the automobile, and security attacks on the \gls{ics} are some
examples\cite{muench2018you}.\newline

Software testing with a combination of manual, automation and exploratory method, is a process to
determine whether a software matches it's predefined specification. It is an informative process
when done in the intended environment. Testing helps in uncovering bugs, failures, unwanted errors and
exceptions\cite{819971}. Fuzzing is an automated technique that can be used to generate and input
random malformed data to the system and observe how the software performs\cite{9787842}.
According to \citeauthor{9787842}, due to the closed nature, complexity and high diversity of the
embedded system, fuzzing is not widely used.\newline

The goal for this thesis was to explore the theories of software testing with fuzzing as a testing
technique. This thesis aims to explain what is fuzzing is, its background and its success over the
years discovering bugs. The aim was to do a proof of concept environment using
Google's OSS-Fuzz\cite{GitHubgo49:online} and clusterfuzzlite\cite{ClusterF90:online} project and
evaluate if fuzzing could be implemented for the embedded systems in the case company.\newline

\section{Background}
This study was carried out for a company in Helsinki, Finland which designs and develops embedded secure
silicon \gls{ip} solutions. The solutions range from Key Management Solutions-which helps in securing
key provisioning Management services, Security protocol solutions- to secure the networking services,
\gls{dpa} solutions- to protect side channels attacks and anti-counter fitting products
to \gls{rot} products-highly secure hardware, software and firmware components. \newline


The \gls{rot} solutions which are best described as isolated and separated from the
software/hardware modules that they are meant to defend against the attackers by performing specific critical
functions\cite{Hardware2:online},\cite{RootsofT78:online}. It is the secure foundation and act as a
safeguard against the attackers for \gls{soc},computer, semiconductor and electronic system.
These products are used in securing supply chain solutions, \gls{iot}, cloud and network and
electronics systems.\newline


The company is certified with \gls{iso}- 9001 standard, which is based on the principles of
continual improvement with a strong customer focus and the top management implication and
motivation\cite{ISOISO9048:online}. With a dedicated quality assurance team, infrastructure, the company
is in constant thrive to improve the system testing processes, standards and ultimately
quality and security of the products. The existing system testing includes testing methods such as, manual,
integration, automation, security, exploratory testing. The company follows the \gls{ci} practices,
where the development, integration and testing are done in an incremental way to detect
the errors early, locating the errors faster and testing more. In recent years fuzzing is becoming a
popular and promising technique for the automation security testing. Therefore, the company is
evaluating different fuzzing tools and techniques for its embedded products which will improve the
quality along with the coverage.\newline


\section{Objectives and Purpose}

The objective and aim of this thesis is to evaluate fuzzing techniques, tools as a software automation
technique to improve the quality, security of the products. This study aimed at gathering information
for following questions:\newline

\begin{itemize}
        \item What is fuzzing in the context of software testing? The introduction.
        \item Overview on fuzzing tools and fuzzers available.
        \item Can automation software testing of embedded application improve by fuzzing?
        \item How to build, choose a good fuzzing target? Which fuzzing tool is the best suitable for
        an embedded project?
\end{itemize}


The scope of this thesis was to evaluate fuzzing techniques and give recommendations based on the result by
doing fuzzing on small program. The aim was to get an idea on how it could be implemented in software automation
testing.\newline


For the future integrations and improvements of fuzzing as a test method in the company,
the results of thesis will be used as a baseline.\newline


\section{Thesis Structure And Outline}

The structure and document outline of this thesis are as follows:\newline
Chapter first introduces the subject, theoretical review and benefits of software testing. The
general explanations on test automation, frameworks. The role of software testing in information security,
theories and background of software testing.\newline

Chapter Two explains about fuzzing, why fuzz?, the past and the present, the history. It presents the success of
fuzzing over the years, the vulnerabilities it has discovered, the bugs it has found. It explains about
fuzzing methods, and the implementations.\newline

Chapter three introduces on the fuzzing tools, different types of fuzzers and their comparisons. It
discusses different open source projects and the ongoing research on fuzzing.\newline

Chapter four introduces about the current state analysis of the embedded project. Based on studies of
the previous chapters choosing appropriate fuzzer for the implementations and proof of concept.\newline

Chapter fifth includes the analysis and results of the implementations and case study.\newline

The rest of the chapters include the conclusions, the evaluations and future studies, along with the references
and appendices.\newline
