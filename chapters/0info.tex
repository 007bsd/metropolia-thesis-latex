\documentclass[12pt,a4paper,oneside,article]{memoir}%Do not touch this first line ;)

% Global information (title of your thesis, your name, degree programme, major, etc.)

\def\bilingual{yes}%For Finnish students, you must have 2 abstracts, one in English and one in your native language (Finnish or Swedish), so "yes", your thesis is bilingual.
%\def\bilingual{no}%For international student writing in English, only one language and one abstract.

\def\thesislang{finnish} %change this depending on the main language of the thesis.
%\def\thesislang{english} % "english" is the only other supported language currently. If someone has the swedish, please contribute!

\def\secondlang{english} %if the main language is Finnish (or Swedish), you must have 2 abstracts (one in Finnish (or Swedish) and one in English)
%\def\secondlang{finnish}
%If the main language is English and that you are native Finnish (or Swedish) speaker, you must have also abstract in your native language on top of the English one.

\author{Nimi/Name} %your first name and last name

%\def\alaotsikko{Alaotsikko/Subtitle} %DISABLED, seems not to be an option with the 2018 template. If you really need it, uncomment and modify style/title.tex accordingly.

%Finnish section, for title/abstract
\def\otsikko{Opinnäytetyön otsikko}
\def\tutkinto{Tutkinto (esim. Insinööri (AMK))} % change to your needs, e.g. "YAMK", etc.
\def\kohjelma{Koulutusohjelma (esim. Tieto\textendash ja viestintätekniikka)}
\def\suuntautumis{Ammatillinen pääaine (esim. Mobile Solutions)}
\def\thesisfi{Insinöörityö}%was Opinnäytetyö
\def\ohjaajat{
Titteli Etunimi Sukunimi\newline
Titteli Etunimi Sukunimi
}
\def\tiivistelma{
Tämä on tiivistelmän ensimmäinen kappale. Tiivistelmän kappaleet loppuvat ii ii ii ii  komentoon newline, jotta saadaan yksi tyhjä rivi aikaiseksi. \newline

Tämä on tiivistlemän toinen kappale.
}
\def\avainsanat{avainsana, avainsana}

%English section, for title/abstract
\title{Your title here}
\def\metropoliadegree{Bachelor of Engineering} % change to your needs, e.g. "master", etc.
\def\metropoliadegreeprogramme{your degree programme (e.g. Information Technology)}
\def\metropoliaspecialisation{your major option (e.g. Mobile Solutions)}
\def\thesisen{Bachelor’s Thesis} % change to your need, e.g. master's
\def\metropoliainstructors{
First name Last name, Title (e.g.: Project Manager)\newline
First name Last name, Title (e.g.: Principal Lecturer)
}
\def\abstract{
Abstract content. To force newline between paragraph in the abstract, you must have both a empty line and the newline command. \newline

beginning of second paragraph\ldots
}
\def\metropoliakeywords{keyword, keyword}
