\documentclass[12pt,a4paper,oneside,article]{memoir}%Do not touch this first line ;)

% Global information (title of your thesis, your name, degree programme, major, etc.)

%\def\bilingual{yes}%For Finnish students, you must have 2 abstracts, one in English and one in your native language (Finnish or Swedish), so "yes", your thesis is bilingual.
\def\bilingual{no}%For international student writing in English, only one language and one abstract.

%\def\thesislang{finnish} %change this depending on the main language of the thesis.
\def\thesislang{english} % "english" is the only other supported language currently. If someone has the swedish, please contribute!

%\def\secondlang{english} %if the main language is Finnish (or Swedish), you must have 2 abstracts (one in Finnish (or Swedish) and one in English)
%\def\secondlang{finnish}
%If the main language is English and that you are native Finnish (or Swedish) speaker, you must have also abstract in your native language on top of the English one.

\author{Bhabani Sankar Das} %your first name and last name

%\def\alaotsikko{Alaotsikko/Subtitle} %DISABLED, seems not to be an option with the 2018 template. If you really need it, uncomment and modify style/title.tex accordingly.

%License
%When publishing your thesis to theseus.fi, you can keep all rights reserved to you,
%or use one of the Creative Commons https://creativecommons.org/licenses/?lang=en
%This attribute will set the license in the metadata of the generated pdf.
%possible options (case sensitive):
%all (keep all rights reserved to yourself)
%by (Attribution)
%by-sa (Attribution-ShareAlike)
%by-nd (Attribution-NoDerivs)
%by-nc (Attribution-NonCommercial)
%by-nc-sa (Attribution-NonCommercial-ShareAlike)
%by-nc-nd (Attribution-NonCommercial-NoDerivs)
%Note that theseus do not accept dedication to public domain CC0
\def\thesiscopy{all}

%Finnish section, for title/abstract
\def\otsikko{Opinnäytetyön otsikko}
\def\tutkinto{Tutkinto (esim. Insinööri (AMK))} % change to your needs, e.g. "YAMK", etc.
\def\kohjelma{Koulutusohjelma (esim. Tieto\textendash ja viestintätekniikka)}
\def\suuntautumis{Ammatillinen pääaine (esim. Mobile Solutions)}
\def\thesisfi{Insinöörityö}%was Opinnäytetyö
\def\ohjaajat{
Titteli Etunimi Sukunimi\newline
Titteli Etunimi Sukunimi
}
\def\tiivistelma{
Tämä on tiivistelmän ensimmäinen kappale. Tiivistelmän kappaleet loppuvat komentoon newline, jotta saadaan yksi tyhjä rivi aikaiseksi. \newline

Tämä on tiivistlemän toinen kappale.
}
\def\avainsanat{avainsana, avainsana}
\def\aihe{Lyhyt kuvaus opinnäytetyöstä. Max 255 merkkiä.}%for the pdf metadata/properties. If not used, empty it and also the \def\subject.

%English section, for title/abstract
\title{Improvement of Embedded System Testing by Adding Fuzzing Technique- A Case Study}
\def\metropoliadegree{Master of Engineering} % change to your needs, e.g. "master", etc.
\def\metropoliadegreeprogramme{Information Technology}
\def\metropoliaspecialisation{Networking and Services }
\def\thesisen{Master’s Thesis} % change to your need, e.g. master's
\def\metropoliainstructors{
First name Last name, Title (e.g.: Project Manager)\newline
First name Last name, Title (Sami Sainio: Principal Lecturer)
}
\def\abstract{
System testing of any software product is very important to maintain good quality, security, and integrity.
Embedded systems which bind the software execution tightly between the hardware architecture and software in a tight
loop has grown at a steady pace over the years. As the embedded systems are complex, it is recommended to apply
software testing throughout the development process.

The manual, traditional and automated testing of these embedded systems often follow the same static approaches to
ncover any vulnerabilities. Fuzzing or fuzz testing is a dynamically automated testing technique which subjects a system to
a stream of input data by modifying the valid seeds to find bugs and vulnerabilities. Fuzzing has been used to exploit
any edge cases and to check the weakness after causing the system to fail.

This thesis explores the possibility of the adoption of fuzzing as a test method to support the embedded software
development lifecycles and improving the system testing and the coverage. The aim is to study different fuzzing
techniques, methods, and the tools available, and doing a case study to implement the best suitable one for the project.
For the implementation of the case study, we studied current state of the art fuzzing tools and techniques and the
classifications of different fuzzing approaches. Machine Learning + fuzzing and side channel aware
fuzzing were not scope of this thesis.

The output and results of the thesis are that fuzzing is a popular way to test the correctness and security of
embedded systems and can be implemented for the improvement of the embedded system testing.
Found out that, the Fuzzing testing must be done continuously. It helps in findings bugs irrespective of the project
requirements and not a replacement to the functional testing but a necessary add-ons.


}
\def\metropoliakeywords{Fuzzing, Embedded system, crypto, software testing}
\def\subject{short description of the thesis. Max 255 characters.}%for the pdf metadata/properties. If not used, empty it and also the \def\aihe.
