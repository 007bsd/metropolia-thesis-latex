
\DocumentMetadata{testphase={phase-III, math}} % comment-out when not using the lualatex-dev
\documentclass[12pt,a4paper,oneside,article]{memoir}%Do not touch this first line ;)

% Global information (title of your thesis, your name, degree programme, major, etc.)

\def\bilingual{no}%For Finnish students, you must have 2 abstracts, one in English and one in your native language (Finnish or Swedish), so "yes", your thesis is bilingual.
%\def\bilingual{no}%For international student writing in English, only one language and one abstract.

\def\thesislang{english} %change this depending on the main language of the thesis.
%\def\thesislang{english} % "english" is the only other supported language currently. If someone has the swedish, please contribute!

%\def\secondlang{english} %if the main language is Finnish (or Swedish), you must have 2 abstracts (one in Finnish (or Swedish) and one in English)
%\def\secondlang{finnish}
%If the main language is English and that you are native Finnish (or Swedish) speaker, you must have also abstract in your native language on top of the English one.

\author{Bhabani Sankar Das} %your first name and last name

%\def\alaotsikko{Alaotsikko/Subtitle} %DISABLED, seems not to be an option with the 2018 template. If you really need it, uncomment and modify style/title.tex accordingly.

%License
%When publishing your thesis to theseus.fi, you can keep all rights reserved to you,
%or use one of the Creative Commons https://creativecommons.org/licenses/?lang=en
%This attribute will set the license in the metadata of the generated pdf.
%possible options (case sensitive):
%all (keep all rights reserved to yourself)
%by (Attribution)
%by-sa (Attribution-ShareAlike)
%by-nd (Attribution-NoDerivs)
%by-nc (Attribution-NonCommercial)
%by-nc-sa (Attribution-NonCommercial-ShareAlike)
%by-nc-nd (Attribution-NonCommercial-NoDerivs)
%Note that theseus do not accept dedication to public domain CC0
\def\thesiscopy{all}

%Finnish section, for title/abstract
\def\otsikko{Opinnäytetyön otsikko}
\def\tutkinto{Tutkinto (esim. Insinööri (AMK))} % change to your needs, e.g. "YAMK", etc.
\def\kohjelma{Koulutusohjelma (esim. Tieto\textendash ja viestintätekniikka)}
\def\suuntautumis{Ammatillinen pääaine (esim. Mobile Solutions)}
\def\thesisfi{Insinöörityö}%was Opinnäytetyö
\def\ohjaajat{
Titteli Etunimi Sukunimi\newline
Titteli Etunimi Sukunimi
}
\def\tiivistelma{
Tämä on tiivistelmän ensimmäinen kappale. Tiivistelmän kappaleet loppuvat komentoon newline, jotta saadaan yksi tyhjä rivi aikaiseksi. \newline

Tämä on tiivistlemän toinen kappale.
}
\def\avainsanat{avainsana, avainsana}
\def\aihe{Lyhyt kuvaus opinnäytetyöstä. Max 255 merkkiä.}%for the pdf metadata/properties. If not used, empty it and also the \def\subject.

%English section, for title/abstract
\title{Securing the Case Company's Embedded Systems: Practical Assessment and Integration of Fuzz Testing Tools}
\def\metropoliadegree{Master of Engineering} % change to your needs, e.g. "master", etc.
\def\metropoliadegreeprogramme{Information Technology}
\def\metropoliaspecialisation{Networking and Services }
\def\thesisen{Master's Thesis} % change to your need, e.g. master's
\def\metropoliainstructors{
Sami Sainio, Senior Lecturer
}
\def\abstract{
% In the era of pervasive digital interconnectedness, the robustness and security
% resilience of embedded software systems have become paramount. These systems
% are integral to various industries, such as automotive, aviation, and consumer
% electronics. As they grow in complexity, they simultaneously present an expanding
% cybersecurity threat landscape. This thesis focuses on enhancing a case company's
% embedded software systems testing methods through fuzz testing-an automated
% technique that uncovers vulnerabilities by inputting invalid or unexpected data.

% %\hspace{2cm}
% \hspace{0.5cm}The motivation for this study stems from the escalating need for
% stronger security measures within embedded systems to counteract growing
% cybersecurity threats. A significant challenge is selecting an appropriate
% fuzzing tool from the multitude of available options. This study, therefore,
% adopts a practical, hands-on approach, conducted within the unique environment
% of a case company.

% \hspace{0.5cm}Following a literature review of various fuzzing tools,
% two popular fuzzing engines-American Fuzzy Lop (AFL++) and libFuzzer,
% were selected for the proof-of-concept.
% These tools were chosen based on their reputation, widespread use,
% and community support. The study examines critical factors such as
% ease of integration, efficiency, speed, code coverage, and ability to
% identify vulnerabilities. This provides an impartial perspective to assist
% the decision-making process.

% \hspace{0.5cm}\hspace{0.5cm}In conclusion, this study offers valuable insights
% into the intricacies of fuzzing, demonstrating the capabilities of tools such as
% AFL++ and libFuzzer. Future research could delve deeper into
% integration of these tools, exploring their scalability and
% effectiveness in CI/CD, and possibly identifying
% areas where new fuzzing techniques or tools might be developed for the case company.

%%%%%% version 5

In today's digital era, ensuring the security and reliability of embedded software systems is
crucial. These systems are integral to various applications and industries, including automotive,
aviation, and consumer electronics. As systems become more complex, they face an increasing number
of cybersecurity threats. This thesis aims to improve the testing methods for embedded software
systems at a specific case company by introducing fuzz testing—an automated technique that
identifies vulnerabilities using unexpected or invalid data.

\hspace{0.5cm}This study follows a practical approach within the company's environment.
A literature review guided the selection of American Fuzzy Lop (AFL++) and libFuzzer for
the proof-of-concept. Factors such as ease of integration, efficiency, and vulnerability
identification were considered in the tool selection process.

\hspace{0.5cm}The decision-making process evaluates AFL++ and libFuzzer's capabilities
and their alignment with the company's software needs and security goals. It also assesses
their feasibility for integration into the software development workflow. Through a careful
assessment of their performance in identifying vulnerabilities using test cases, the company
can decide whether to integrate one or both tools to enhance software security through automated fuzz testing.

\hspace{0.5cm}In conclusion, this study gives a basic summary of how fuzz
testing works and shows what AFL++ and libFuzzer can do. Future research could explore their
integration into CI/CD pipelines, scalability, and real-world effectiveness. There is potential
for further exploration of fuzzing techniques or tools tailored to the cybersecurity needs of the
case company.
}
\def\metropoliakeywords{fuzzing, embedded system, AFL++, libFuzzer}
\def\subject{short description of the thesis. Max 255 characters.}%for the pdf metadata/properties. If not used, empty it and also the \def\aihe.
