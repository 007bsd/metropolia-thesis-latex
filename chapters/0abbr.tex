% Generate the glossary
\makeglossaries

% Acronyms, abbreviations, etc.

\newacronym{html}{HTML}{HyperText Markup Language}
\newacronym{sql}{SQL}{Structured Query Language}
\newacronym{io}{I/O}{Input/Output}
\newacronym{ram}{RAM}{Random Access Memory}
\newacronym{php}{PHP}{Hypertext Preprocessor}
\newacronym{wysiwym}{WYSIWYM}{What You See Is What You Mean}
\newacronym{isbn}{ISBN}{International Standard Book Number}
\newacronym{url}{URL}{Uniform Resource Locator}
\newacronym{doi}{DOI}{Digital Object Identifier}
\newacronym{iot}{IoT}{Internet of Things}
\newacronym{ics}{ICS}{Industrial Control System}
\newacronym{ip}{IP}{Intellectual Property}
\newacronym{rot}{ROT}{Root of Trust}
\newacronym{soc}{SoC}{system-on-chip}
\newacronym{os}{OS}{Operating Systems}
\newacronym{iso}{ISO}{International Organization of Standardization}
\newacronym{ci}{CI}{Continuous Integration}
\newacronym{dpa}{DPA}{Differential Power Analysis}

% Glossary entries

\newglossaryentry{part_key}{
	name={partition key},
	description={a column or set of columns from the same table whose consolidated value decide the partition for a given data}
}
\newglossaryentry{thesis}{
	name=thesis,
	description={a written essay one submitted for a university degree},
	plural=theses
}
\newglossaryentry{latex}
{
	name=\LaTeX{},
	description={Is a mark up language specially suited for scientific documents}
}

\newglossaryentry{maths}
{
	name=mathematics,
	description={Mathematics is what mathematicians do}
}


