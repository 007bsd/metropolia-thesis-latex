% Conclusions
%\clearpage%if the chapter heading starts close to bottom of the page, force a line break and remove the vertical vspace
\vspace{21.5pt}
\chapter{Discussions and Conclusions}

\section{Discussions}
Fuzzing, in the realm of software testing, is multifaceted. At its core,
fuzzing can be relatively straightforward to deploy, acting as a quick initial
sieve for potential vulnerabilities. However, in other contexts, its
implementation demands extensive groundwork, especially when targeting
complex or novel systems. The findings from fuzzing may vary – while some
vulnerabilities are promptly detected, others remain elusive. One inherent
limitation of fuzzing is its inability to conclusively determine a system's robustness.
The absence of detected bugs doesn't necessarily indicate a system's invulnerability;
instead, it could highlight areas where the fuzzer might require optimization or redirection.

The landscape of fuzzing is dynamic. Although numerous open-source fuzzers
continuously emerge, often enhancing specific components of fuzzing,
their evolution is twofold. Some lose traction, lacking the maintenance to
remain relevant. In contrast, others merge into larger projects
contributing their innovations to more extensive frameworks.
A notable examples in the realm of modern fuzzing tools are AFL++, libFuzzer, Atheris and other
custom fuzzers which exemplify the principle of continuous adaptation and improvement
in the fuzzing community.

The dawn of AI-enhanced fuzzing\cite{GoogleOn27:online} introduces another layer of complexity.
AI-driven fuzzing tools, leveraging machine learning models,
promise adaptive test scenarios and heightened vulnerability detection rates.
They aim to overcome the traditional limitations of fuzzers, offering the potential
for more context-aware and target-specific vulnerability detection.
However, while promising, this AI infusion is still in initial stages,
requiring comprehensive evaluation over time.

Reflecting on this journey of exploring fuzzing, it became evident
that the field, although a sub-discipline of software testing,
offers a vast expanse of knowledge. Venturing into this domain presented
numerous challenges, from grasping foundational concepts to navigating
intricate tooling setups. Notably, substantial time was invested in comprehending
the basics and configuring systems aptly for fuzzing. With this foundational
understanding now in place, future endeavors can pivot towards assessing varied
fuzzing methodologies across diverse scenarios.

In conclusion, while fuzzing provides an invaluable avenue for enhancing system
security, the path forward involves continuous learning, tool assessment,
and methodological refinement. As the field evolves, especially with AI's inclusion,
it's imperative to stay updated, prioritizing objectives to harness the maximum
potential of fuzzing in software testing.

\section{Conclusions}
This research highlights the significant potential of fuzzing within software
testing for embedded systems. Hands-on experimentation using both
libFuzzer and AFL++ within the embedded system architecture of the case company
demonstrated their effectiveness and ability to swiftly identify vulnerabilities.
The study showed that these fuzzing tools can detect issues more rapidly than
certain traditional methods.

A notable aspect of these fuzzing tools is the incorporation of built-in
reporting mechanisms. This functionality demystifies the intricate task of
determining and comprehending code coverage, assuring that potential
vulnerabilities are duly identified. The automation and clarity these tools
provide can significantly enhance the efficiency of the testing process.

Looking ahead, there exists a strong rationale for deeper integration of
fuzzing into standard software development lifecycles, especially within
continuous testing frameworks. As the intricacy of software escalates, so does
the need for thorough and streamlined testing processes. The adoption of
fuzzing tools, as evidenced by the outcomes using libFuzzer and AFL++, presents
a tactical advantage for organizations, guaranteeing software that is both
operational and resilient to potential threats.

To conclude, this thesis, while presenting an overview of the prevailing
scenario, also serves as a roadmap for the case company's onward
testing strategies. The importance of persistently evolving testing
methodologies is accentuated, and with the knowledge derived from
the empirical tests, the company stands poised for the adoption of more
sophisticated and integrated testing approaches.