% % Appendix
% % And demonstrate text references and bibliography references in appendix


\chapter{Test Data Creation for AFL++-QEMU mode}\label{appx:third}
The following python program to create test data for AFL++-QEMU mode.

\section*{generate\_files.py}
\begin{minted}{python}
import os
import random
import secrets
import struct

# Number of files to generate
num_files = 10

# Size of the buffer in bytes
buffer_size = 1024

for i in range(num_files):

    # Generate a random integer
    random_int = random.randint(-2147483648, 2147483647)
    int_str = str(random_int)

    # Generate a random float
    random_float = random.uniform(-1e5, 1e5)
    float_str = f"{random_float:.2f}"

    # Generate a random string
    random_string = secrets.token_urlsafe(random.randint(1, 50))

    # Generate random bytes
    random_bytes = os.urandom(random.randint(1, 50))

    # Generate a random boolean
    random_bool = random.choice([True, False])
    bool_str = str(random_bool)

    # Combine all data into one string
    combined_data = f"{int_str}\n{float_str}\n{random_string}\n{random_bytes}\n{bool_str}\n".encode('utf-8')

    # Check buffer size and truncate if necessary
    if len(combined_data) > buffer_size:
        combined_data = combined_data[:buffer_size]

    # Create a file and write the data to it
    with open(f"input/file_{i}.bin", "wb") as f:
        f.write(combined_data)

\end{minted}

%\clearpage