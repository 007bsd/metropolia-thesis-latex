% Appendix
% And demonstrate text references and bibliography references in appendix

\chapter{libFuzzer Target Program}~\label{appx:second}
The following is the detailed implementation of the
target program~\cite{FuzzingE54:online}~\cite{zephyrsa35:online} used in
the fuzz testing with Zephyr apps by utilizing the libFuzzer:

\section*{main.c}
\begin{minted}[linenos,frame=lines,baselinestretch=1.2,breaklines]{c}
#include <stdio.h>
#include <stdbool.h>
#include <stdint.h>


/* Fuzz testing is coverage-based, so we want to hide a failure case
 * (a write through a null pointer in this case) down inside a call
 * tree in such a way that it would be very unlikely to be found by
 * randomly-selected input.  But the fuzzer can still find it in
 * linear(-ish) time by discovering each new function along the way
 * and then probing that new space.  The 1 in 2^56 case here would
 * require months-to-years of work for a large datacenter, but the
 * fuzzer gets it in 20 seconds or so. This requires that the code for
 * each case be distinguishable/instrumentable though, which is why we
 * generate the recursive handler functions this way and disable
 * inlining to prevent optimization.
 *
 * https://github.com/zephyrproject-rtos/zephyr/blob/main/samples/subsys/debug/fuzz/src/main.c
 */

int *global_null_ptr;

// crash
static const uint8_t key[] = { 0x9e, 0x21, 0x0c, 0x18, 0x9d, 0xd1, 0x7d };
bool found[sizeof(key)/sizeof(key[0])];

#define LASTKEY (sizeof(key)/sizeof(key[0]) - 1)

#define GEN_CHECK(cur, nxt)                           \
    void check##nxt(uint8_t *data, size_t sz);        \
    void check##cur(uint8_t *data, size_t sz)         \
    {                                                 \
        if (cur < sz && data[cur] == key[cur]) {      \
            if (!found[cur]) {                        \
                printf("Found key %d\n", cur);        \
                found[cur] = true;                    \
            }                                         \
            if (cur == LASTKEY) {                     \
                *global_null_ptr = 0; /* boom! */     \
            } else {                                  \
                check##nxt(data, sz);                 \
            }                                         \
        }                                             \
    }

GEN_CHECK(0, 1)
GEN_CHECK(1, 2)
GEN_CHECK(2, 3)
GEN_CHECK(3, 4)
GEN_CHECK(4, 5)
GEN_CHECK(5, 6)
GEN_CHECK(6, 0)

#ifdef USE_LIBFUZZER
int LLVMFuzzerTestOneInput(const uint8_t *data, size_t size) {
    check0((uint8_t*)data, size);
    return 0;
}
#else
int main(void)
{
    /*Matches key
    causes crash*/
    //uint8_t test_data[] = { 0x9e, 0x21, 0x0c, 0x18, 0x9d, 0xd1, 0x7d };
    /*Does not match key
    does not crash*/
    uint8_t test_data[] = { 0x9e, 0x21, 0x0c, 0x18, 0x9d, 0xd1, 0x7e };
    check0(test_data, sizeof(test_data)/sizeof(test_data[0]));
    return 0;
}
#endif

\end{minted}

\section*{CMakeLists.txt}
\begin{minted}{cmake}
    cmake_minimum_required(VERSION 3.20.0)
    project(sample_zephyr_libfuzzer)

    option(USE_LIBFUZZER "Build with libfuzzer" OFF)

    set(CMAKE_C_COMPILER clang)

    if(USE_LIBFUZZER)
      add_definitions(-DUSE_LIBFUZZER)
      set(CMAKE_C_FLAGS "${CMAKE_C_FLAGS} -g -fsanitize=fuzzer,address,undefined")
      set(CMAKE_EXE_LINKER_FLAGS "${CMAKE_EXE_LINKER_FLAGS} -g -fsanitize=fuzzer,address,undefined")
      set(CMAKE_C_FLAGS "${CMAKE_C_FLAGS} -g -fprofile-instr-generate -fcoverage-mapping")
    endif()

    add_executable(zephyr_libfuzzer src/main.c)

    if(USE_LIBFUZZER)
      set_target_properties(zephyr_libfuzzer PROPERTIES LINK_FLAGS "-g -fsanitize=fuzzer,address,undefined")
    endif()

\end{minted}
\clearpage