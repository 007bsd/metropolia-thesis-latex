% Appendix
% And demonstrate text references and bibliography references in appendix
\vspace{21.5pt}
\clearpage
\chapter{AFL++ Target Program}~\label{appx:first}
The following is the detailed implementation of the
target program~\cite{FuzzingE54:online}~\cite{zephyrsa35:online} used in
the fuzz testing with Zephyr apps by utilizing the AFL++~\cite{257204}:

The below given is the target program for the AFL++.
\section*{main.c}

\begin{minted}[linenos,frame=lines,baselinestretch=1.2,breaklines]{c}
#include <stdio.h>
#include <stdbool.h>
#include <stdint.h>
#include <stdlib.h>

/* Fuzz testing is coverage-based, so we want to hide a failure case
 * (a write through a null pointer in this case) down inside a call
 * tree in such a way that it would be very unlikely to be found by
 * randomly-selected input.  But the fuzzer can still find it in
 * linear(-ish) time by discovering each new function along the way
 * and then probing that new space. The 1 in 2^56 case here would
 * require months-to-years of work for a large datacenter, but the
 * fuzzer gets it in 20 seconds or so. This requires that the code for
 * each case be distinguishable/instrumentable though, which is why we
 * generate the recursive handler functions this way and disable
 * inlining to prevent optimization.
 *
 * https://github.com/zephyrproject-rtos/zephyr/blob/main/samples/subsys/debug/fuzz/src/main.c
 */

int *global_null_ptr;

// crash
static const uint8_t key[] = { 0x9e, 0x21, 0x0c, 0x18, 0x9d, 0xd1, 0x7d };
bool found[sizeof(key)/sizeof(key[0])];

#define LASTKEY (sizeof(key)/sizeof(key[0]) - 1)

#define GEN_CHECK(cur, nxt)                           \
    void check##nxt(uint8_t *data, size_t sz);        \
    void check##cur(uint8_t *data, size_t sz)         \
    {                                                 \
        if (cur < sz && data[cur] == key[cur]) {      \
            if (!found[cur]) {                        \
                printf("Found key %d\n", cur);        \
                found[cur] = true;                    \
            }                                         \
            if (cur == LASTKEY) {                     \
                *global_null_ptr = 0; /* boom! */     \
            } else {                                  \
                check##nxt(data, sz);                 \
            }                                         \
        }                                             \
    }

GEN_CHECK(0, 1)
GEN_CHECK(1, 2)
GEN_CHECK(2, 3)
GEN_CHECK(3, 4)
GEN_CHECK(4, 5)
GEN_CHECK(5, 6)
GEN_CHECK(6, 0)


int main(int argc, char **argv)
{
    /*Matches key
    causes crash*/
    //uint8_t test_data[] = { 0x9e, 0x21, 0x0c, 0x18, 0x9d, 0xd1, 0x7d };
    // no crash
    //uint8_t test_data[] = { 0x9e, 0x21, 0x0c, 0x18, 0x9d, 0xd1, 0x7e };
    // Open the file
   if (argc != 2) {
        printf("Usage: %s <input_file>\n", argv[0]);
        return 1;
    }

    // Open the file
    FILE *file = fopen(argv[1], "rb");
    if (file == NULL) {
        printf("Could not open file %s\n", argv[1]);
        return 1;
    }

    // Determine the size of the file
    fseek(file, 0, SEEK_END);
    size_t size = ftell(file);
    fseek(file, 0, SEEK_SET);

    // Allocate memory for the data
    uint8_t *test_data = malloc(size);
    if (test_data == NULL) {
        printf("Could not allocate memory for test data\n");
        fclose(file);
        return 1;
    }

    // Read the data from the file
    if (fread(test_data, 1, size, file) != size) {
        printf("Error reading file %s\n", argv[1]);
        free(test_data);
        fclose(file);
        return 1;
    }

    // Close the file
    fclose(file);

    // Use the data
    check0(test_data, size);

    // Clean up
    free(test_data);
    return 0;
}

\end{minted}
\captionof{listing}{main.c of AFL++ target}\label{lst:main_program_afl}

\pagebreak

The below is the \textit{CMakeLists.txt} use for compiling the
AFL++ target program.
\section*{CMakeLists.txt}
\begin{minted}[linenos,frame=lines,baselinestretch=1.2,breaklines]{cmake}
    cmake_minimum_required(VERSION 3.20.0)
    project(sample_zephyr_afl)

    option(USE_AFL "Use afl" OFF)

    if(USE_AFL)
        set(CMAKE_C_COMPILER afl-gcc-fast)
        set(CMAKE_C_FLAGS "${CMAKE_C_FLAGS} -g -fpic -fprofile-arcs -ftest-coverage")
    else()
        set(CMAKE_C_COMPILER gcc)
    endif()

    add_executable(zephyr_aflfuzzer src/main.c)
    \end{minted}
    \captionof{listing}{CMakeLists.txt of AFL++}\label{lst:cmake_list_txt_afl}